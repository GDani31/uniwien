\section{Was für ein Tiger?}

Schauen wir uns zunächst das vorgegebene Abstract\footnote{\url{https://arxiv.org/pdf/1906.05047}} an. Dabei sind alle Punkte markiert, die ich weiter recherchieren werde.

Wir stellen Tiger vor, ein Brillensystem, das Benutzern dabei hilft, die \textbf{20-20-20-Regel} anzuwenden, um die Symptome des \textbf{Computer Vision Syndroms} zu lindern. Es überwacht die Bildschirmaktivitäten des Benutzers und bietet Echtzeit-Feedback, um ihnen zu helfen, die Regel einzuhalten. Für eine präzise Erkennung der Bildschirmansicht entwickeln wir einen leichten multisensorischen Ansatz mit drei Sensorikmodalitäten: Farbe, \textbf{IMU} und \textbf{Lidar}. Außerdem gestalten wir das Echtzeit-Feedback, um Benutzer effektiv dabei zu unterstützen, die Regel einzuhalten. Unsere Evaluation zeigt, dass Tiger Bildschirmansichtsereignisse genau erkennt und unempfindlich gegenüber Unterschieden in Bildschirmtypen, Inhalten und Umgebungslicht ist. Unsere Benutzerstudie zeigt eine positive Wahrnehmung von Tiger hinsichtlich seiner Nützlichkeit, Akzeptanz und Echtzeit-Feedback.

Um weitere Papers zu finden, verwende ich Google Scholar mit meinem SearXNG als Proxy. Alle Quellen (außer anderweitig angegeben) finde ich über diese Methode.

\section{Recherche}

\subsection{20-20-was}

Zuerst war ich unsicher bezüglich des ersten Begriffs. Von einer "20-20-20-Regel" hatte ich bisher noch nichts gehört, deshalb werde ich mir dazu Literatur\footnote{\url{https://journals.lww.com/ijo/fulltext/2023/05000/The_20_20_20_rule__Practicing_pattern_and.71.aspx}} anschauen.

Hier habe ich herausgefunden, dass die 20-20-20-Regel besagt, dass man alle 20 Minuten für 20 Sekunden mindestens 20 Fuß (ungefähr 6 Meter) entfernt in die Ferne schauen soll. Weiterhin bin ich auf Symptome gestoßen, die durch die 20-20-20-Regel verhindert oder zumindest gelindert werden können.

\begin{itemize}
  \item doppeltes Sehen
  \item juckende oder gereizte Augen
  \item Schwindel oder Kopfschmerzen
\end{itemize}

Ich werde nun herausfinden, ob diese Symptome teilweise dem \textbf{Computer Vision Syndrome} entsprechen.

\subsection{Alptraum eines jeden Informatikers}

Es scheint also tatsächlich ein "Computer Vision Syndrome" zu geben. Für jemanden, der hauptberuflich Bildschirme betrachtet, ist das beunruhigend. Allein nach dem Abstract dieses Papers\footnote{\url{https://www.sciencedirect.com/science/article/abs/pii/S0039625705000093}} zeigt sich, dass es sich um ähnliche Symptome wie bei der 20-20-20-Regel handelt. Alle möglichen Variationen dieser Symptome werden als Computer Vision Syndrome bezeichnet und treten offenbar bei vielen Erwachsenen auf.

\subsection{Aktiv - eher nicht}

Das nächste, was mir aufgefallen ist, sind die \textbf{screen viewing activities} und ob es dabei mehr Unterscheidungen gibt als nur "Bildschirm betrachten".

Zuerst konnte ich nicht genau finden, wonach ich suchte, da ich fast ausschließlich Papers über Kinder, Jugendliche und junge Erwachsene fand. Aber ich habe ein neues Wort gelernt: "sedentär" - vor einem Bildschirm. Auch einiges bezüglich Übergewichtsproblemen\footnote{\url{https://link.springer.com/article/10.1186/1471-2458-10-593}} und Verhaltensstörungen\footnote{\url{https://publications.aap.org/pediatrics/article-abstract/126/5/e1011/65363/Children-s-Screen-Viewing-is-Related-to}} ist zu finden.

Daher nehme ich an, dass es bei diesem Begriff nicht um genauere Unterscheidungen geht, sondern eher um einen Oberbegriff für jegliche Bildschirmzeit.

\subsection{IMU}

IMU steht für Inertial Measurement Unit. Diese kombiniert Gyroskop, Beschleunigungsmesser und Magnetometer zu einer Einheit, der IMU.
Diese werden vielseitig verwendet, beispielsweise bei der Analyse von Gelenksbewegungen\footnote{\url{https://www.mdpi.com/1424-8220/14/4/6891}} in der Sportmedizin. Ich nehme mal stark an, dass es sich bei Motion Capture auch um diese Technologie handelt.

\section{Spekulazius}

\subsection{Augenware}

Im Abstract des ursprünglichen Papers wird von \textbf{eyewear} gesprochen. Ich nehme an, es handelt sich um eine Art Eye-Tracking-Brille, mit der die Augenbewegungen festgehalten werden können. Dadurch sind vermutlich einige Symptome des Computer Vision Syndroms erkennbar/messbar. Mit den IMU-Sensoren werden vermutlich Bewegungen von Kopf und Körper gemessen (der sich kaum bewegt, wenn man auf einen Bildschirm schaut), und mit Lidar wird wahrscheinlich der Abstand zum Bildschirm oder anderen Objekten gemessen.

\subsection{20-20-20 Einhalten}

Da das Ziel das Einhalten der 20-20-20-Regel ist, wird vermutlich auf erste Anzeichen des Computer Vision Syndroms geachtet. Vermutlich treten nach längeren Zeiten vor dem Bildschirm die Symptome selbst bei Einhaltung der Regel früher oder öfter als alle 20 Minuten auf.
Demnach kann gewarnt oder erinnert werden, die 20-20-20-Regel jetzt anzuwenden und eine kurze Pause einzulegen.

Es wird auch möglich sein, den Effekt der Pause besser zu messen und gegebenenfalls die Pause länger als nur 20 Sekunden zu gestalten, sollten alleine 20 Sekunden nicht ausreichen. Hier kommt mir gerade in den Sinn, dass Lidar zusammen mit Eye-Tracking auch dafür verwendet werden kann, um herauszufinden, auf welche Objekte in der Pause geschaut wird, und demnach ob die mindestens 7 Meter Entfernung eingehalten wird.

\subsection{Aber bitte konstruktiv!}

Wie

 könnte solch ein Echtzeit-Feedback aussehen? In primitivster Form eine teure Eieruhr, die alle 20 Minuten an die Pause erinnert. Weiterhin wird Feedback dazu gegeben, wenn der Abstand zum Bildschirm zu gering ist oder die Helligkeit entsprechend der Umgebungsbeleuchtung zu hoch ist. Sollten Anzeichen eines Computer Vision Syndroms auftreten, kann auch schon früher als alle 20 Minuten eine Pause eingelegt werden, und die Pause kann auch durch die Brille verifiziert und gegebenenfalls verlängert werden.

\section{Konklusio}

Das Ziel ist es, eine sichere Möglichkeit für den Benutzer zu bieten, um die 20-20-20-Regel einzuhalten. Dabei soll eine modifizierte Eye-Tracking-Brille verwendet werden, um zu erkennen, wann auf einen Bildschirm geschaut wird und jeweils zur richtigen Zeit eine Pause anzuzeigen.

\newpage

\section{Appendix 1 - So verschieden}

\subsection{Alarm!}

Einen Punkt, den ich überhaupt nicht bedacht habe, ist, wie das Feedback an den Benutzer gegeben wird. Die Implementierung verwendet einen Vibrationsmotor und LED-Lichter, die der Benutzer sehen kann, um an eine Pause zu erinnern.

\subsection{Inhaltliche Fragen}

Ebenso habe ich nicht bedacht, dass ein Bildschirm mehr kann als nur Bilder anzeigen, sondern auch bewegte Bilder\footnote{ein wirklich radikaler Gedanke, muss ich zugeben} in Form von Videos oder Spielen.

\section{Appendix 2 - Und doch so gleich}

\subsection{Implementierung}

Meine Gedanken zu einer Eye-Tracking-Brille waren vollkommen richtig. Es handelt sich um einen Raspberry Pi Zero auf einer Brille mit den angesprochenen Sensoren. Der Farbsensor erkennt die unterschiedlichen Lichtverhältnisse von digitalen Bildschirmen, mit dem IMU wird die Bewegung des Kopfes gemessen (der sich kaum bewegt, wenn man auf einen Bildschirm schaut), und mit Lidar wird der Abstand zum Bildschirm oder anderen Objekten gemessen.
