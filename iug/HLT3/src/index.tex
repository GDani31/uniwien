
\section{Was fuer ein Tiger?}

Schauen wir uns zunaechst das vorgegebene Abstract\footnote{\url{https://arxiv.org/pdf/1906.05047}}
an. Dabei sind alle diejenigen Punkte markiert, die ich weiter recherchieren werde.

We propose Tiger, an eyewear system for helping users follow the \textbf{20-20-20 rule} to alleviate the \textbf{Computer Vision Syndrome} symptoms. It monitors user’s \textbf{screen viewing activities} and provides real-time feedback to help users follow the rule. For accurate screen viewing detection, we devise a light-weight multi-sensory fusion approach with three sensing modalities, color, \textbf{IMU}, and \textbf{lidar}. We also \textbf{design the real-time feedback} to effectively lead users to follow the rule. Our evaluation shows that Tiger accurately detects screen viewing events, and is robust to the differences in screen types, contents, and ambient light. Our user study shows positive perception of Tiger regarding its usefulness, acceptance, and real-time feedback.

Um weitere Papers zu finden, verwende ich Google Scholar mit meinem SearXNG als Proxy. Alle Quellen (ausser anderweitig angegeben) finde ich ueber diese Methode.

\section{Recherche}

\subsection{20-20-was}

Das erste ist der erste Term mit dem ich nichts anfangen hab koennen. Von einer "20-20-20 Regel" habe ich bisher nichts zu tun gehabt, deswegen werde 
ich mir dazu etwas Literatur\footnote{\url{https://journals.lww.com/ijo/fulltext/2023/05000/The_20_20_20_rule__Practicing_pattern_and.71.aspx}} anschauen.

Hier habe ich herausgefunden, dass die 20-20-20 Regel beschreibt nach 20 Minuten Bildschirmzeit 20 Sekunden in die Ferne zu schauen (>20 Fuss Entfernung). 
Weiters bin ich auf Symptome gestossen, die durch die 20-20-20 Regel verhindert oder zumindest gelindert werden koennen.

\newpage

\begin{itemize}
  \item doppelte Sicht
  \item juckende oder gereizte Augen
  \item Schwindel oder Kopfschmerzen
\end{itemize}

Ob es sich dabei - zumindest teilweise - 
um das \textbf{Computer Vision Syndrome} handelt, werde ich jetzt herausfinden

\subsection{Alptraum eines jeden Informatikers}

Jetzt soll es also ein "Computer Vision Syndrome" geben. Jemanden der hauptberuflich auf Bildschirme starrt\footnote{Der Bug findet sich nicht von selbst!}
macht das leicht nervoes. Allein nach dem Abstract dieses Papers\footnote{\url{https://www.sciencedirect.com/science/article/abs/pii/S0039625705000093}}
stellt sich heraus, dass es sich tatsaechlich um aehnliche Symptome handelt, wie zuvor bereits bei der Quelle der 20-20-20 Regel. 
Alle moeglichen Variationen dieser Symptome werden als Computer Vision Syndrome bezeichnet und treten offenbar bei vielen Erwachsenen auf.

\subsection{Aktiv - eher nicht}

Das naechste was mir ins Auge gesprungen ist sind die \textbf{screen viewing activities} und ob es da mehr Unterscheidungen gibt als "schaut auf Bildschirm".

Zunaechst kann ich nicht genau finden, wonach ich suche. Denn ich finde fast ausschliesslich Paper ueber Kinder, Jugendliche und junge Erwachsene.
Aber ich habe ein neues Wort gelernt: "sedentary" - vor einem Bildschirm. Auch einiges bezueglich 
Uebergewicht\footnote{\url{https://link.springer.com/article/10.1186/1471-2458-10-593}}
und Verhaltensstoerungen\footnote{\url{https://publications.aap.org/pediatrics/article-abstract/126/5/e1011/65363/Children-s-Screen-Viewing-is-Related-to}}
ist zu finden.

Daher nehme ich an, dass sich bei diesem Wortlaut nicht um genauere Unterscheidungen geht, sondern mehr ein Ueberbegriff fuer jegliche Bildschirmzeit ist.

\subsection{IH EM U}

IMU steht fuer Inertial Measurement Unit. Dieses kombiniert Gyroskop, Beschleunigungsmeter und Magnetometer zu einer Einheit, der IMU.
Diese werden vielseitig verwendet, beispielsweise bei der Analyse von Gelenksbewegungen\footnote{\url{https://www.mdpi.com/1424-8220/14/4/6891}}
in der Sportmedizin. Ich nehme mal stark an, dass es sich bei Motioncapture auch um diese Technologie handelt.

\section{Spekulazius}

\subsection{Augenware}

Im Abstract des urspruenglichen Papers wird von \textbf{eyewear} gesprochen. Ich nehme an, es wird eine Art Eye-Tracking Brille sein, mit der die
Augenbewegungen festgehalten werden. Dadurch sind vermutlich einige Symptome des Computer Vision Syndrome erkennbar/messbar. Mit den IMU Sensoren
werden vermutlich Bewegungen von Kopf und Koerper gemessen. Lidar kann ich mir nur vorstellen als Tracking der Umgebung, 
beispielsweise Abstand zum Bildschirm, wobei dafuer denke ich billigere Sensoren genauso gut geeignet waeren.

\subsection{20-20-20 Einhalten}

Nachdem das Ziel das Einhalten der 20-20-20 Regel ist, wird vermutlich auf erste Anzeichen von Computer Vision Syndrome geachtet. 
Vermutlich treten nach laengeren Zeiten vorm Bildschirm die Symptome selbst bei Einhaltung der Regel frueher oder oefter als alle 20 Minuten auf.
Demnach kann gewarnt beziehungsweise erinnert werden, die 20-20-20 Regel jetzt anzuwenden und eine kurze Pause einzulegen.

Genauso wird es auch moeglich sein den Effekt der Pause besser zu messen und gegebenfalls die Pause laenger als nur 20 Sekunden
zu gestalten, sollten alleine 20 Sekunden nicht ausreichen. Hier kommt mir gerade in den Sinn, dass Lidar gemeinsam mit Eye-Tracking auch 
dafuer werden kann, um herauszufinden auf welche Objekte geschaut wird in der Pause und demnach ob die mindestens 7 Meter Entfernung 
eingehalten werden.

\subsection{Aber bitte konstruktiv!}

Wie koennte so ein real-time feedback aussehen? In primitivster Form eine teure Eieruhr, die alle 20 Minuten an die Pause erinnert.
Weiters wird es Feedback dazu geben, wenn der Abstand zum Bildschirm zu gering ist oder die Helligkeit entsprechend der Umgebungsbeleuchtung
zu hoch ist. Sollten Anzeichen eines Computer Vision Syndrome auftreten kann auch schon frueher als 20 Minuten eine Pause eingelegt werden
und die Pause kann auch durch die Brille verifiziert und gegenbenenfalls verlaengert werden.

\section{Konklusio}

Ziel ist es eine sichere Moeglichkeit dem Benutzer zu bieten, um die 20-20-20 Regel einzuhalten. Dabei soll eine modifizierte Eye-Tracking Brille verwendet werden um zu erkennen, wann auf einen Bildschirm geschaut wird und jeweils zur richtigen Zeit eine Pause ankuendigt.

\section{Appendix 1 - So verschieden}

\subsection{Alarm!}

Einen Punkt den ich ueberhaupt nicht bedacht habe ist wie das Feedback an den Benutzer gegeben wird. Die Implementation benutzt 
einen Vibrationsmotor und LED Lichter, die der Benutzer sehen kann, um an eine Pause zu erinnern. 

\subsection{Inhaltliche Fragen}

Genauso habe ich nicht bedacht, dass ein Bildschirm mehr kann als nur Bilder anzeigen, sondern auch noch bewegte Bilder\footnote{ein wirklich radikaler Gedanken, muss ich zugeben} in Form von Videos oder Spielen.

\section{Appendix 2 - Und doch so gleich}

\subsection{Implementierung}

Mein Gedanken mit einer Eye-Tracking Brille war vollkommen richtig. Es ist ein Raspberry Pi Zero auf einer Brille mit den angesprochenen 
Sensoren. Der Farbsensor erkennt die unterschieden Lichtverhaeltnisse von digitalen Bildschirmen, mit dem IMU wird die Bewegung des Kopfes gemessen (der sich kaum bewegt, wenn man auf einen Bildschirm schaut) und mit Lidar der Abstund zum Bildschirm oder anderen Objekten. 

